\section{Relevant literature}
While there are not many resources focusing specifically on Twitch and the way it treats user data, the entire field of critical data studies is currently on the rising edge of the modern society. Numerous books, articles and other information sources have emerged studying the social impact of exponentially growing data accumulation. The process of collecting data from increasingly many areas of human life and of transforming this information into economic value is called \textit{datafication}. 

Some of the sources relevant for this research take an optimistic approach to the matter, like for instance M. Salganik (2016)~\cite{seminar-salganik}. In his book on the evolution of social research he states that, despite being a social scientist himself, he is willing ``to adopt the optimistic tone of a data scientist"~\cite{seminar-salganik}. In his work, he tries to abstain from using field-specific terminology in order to target readers who work on the narrow verge between social science and data analytics. Thus, the book is advisable as a read for interdisciplinary researchers. 

Another author taking more of a negative approach while writing specifically on the topic of corporate surveillance is Wolfie Christl (2017)~\cite{seminar-christl}. ``With the actors guided only by economic goals, a data environment has emerged in which individuals are constantly surveyed and evaluated, categorized and grouped, rated and ranked, numbered and quantified, included or excluded, and, as a result, treated differently''~\cite{seminar-christl} - this is how the digital rights activist based in Vienna, sees the current development in the area of data collection. One of the leading questions he tries to answer in his article is "How do online platforms, tech companies, and data brokers collect, trade, and make use of personal data?"~\cite{seminar-christl}. The train of thought that he follows is extensively visualized by various graphical representations of data flow between large data-driven companies, such as Google and Facebook, ever-present data brokers, and other entities taking part in the processes of data gathering and data analytics. 

For the readers looking for a more scientific approach to the implications of datafication ``Critical questions for big data''~\cite{seminar-crawford} by K. Crawford et al. (2012) can be a good starting point. The authors of this scientific article offer ``six provocations to spark conversations about the issues of Big Data''~\cite{seminar-crawford}. These 'provocations' concentrate on the areas of technology, analysis and mythology, and are expected to create a lively discussion with the reader, while focusing on both utopian and dystopian aspects of datafication. 

Furthermore, one of the few Twitch-centered resources available on the web is the official Data Science Blog~\cite{twitch-data-blog} based on the company's own platform. It provides numerous entries and articles on the various aspects of what and how Twitch does with the massive data sets it collects. Some of the entries are of profoundly technical nature and focus on the practical aspects of what can be done with the acquired data. Others, however, include sociological elements and sometimes even spark a discussion between blog followers. 
