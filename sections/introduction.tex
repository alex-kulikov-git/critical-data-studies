\section{Introduction} 
Humanity has always been obsessed with observing, measuring and analyzing
its environment. We have built sophisticated instruments, equipped with
sensors, to gather data. There are metrics one can measure directly, such
as distance or voltage, while other units must be inferred from a number
of assessments. We tend to measure and deduct from very feasible heuristics.
Usually, the more one observes, the greater and more accurate the conclusion 
becomes. In our time and age, utilizing the strong computational 
capabilities at our disposal, we even dare to predict human behavior. 
The starting point for such predictions is data, which nowadays can be easily collected online.
The Internet is foremost a web of servers and clients, connected through 
coaxial cables, wireless transmitters and receivers. Apart from some 
autonomous systems, the web is a passive medium. This means that individuals decide all by themselves which website they want to visit. 
This decision making is exactly what data collection on the Internet is about --- human behavior. Because we are dealing with a 
passive medium, behind every click sits a living person, which has the monetary
means to acquire various products. Data scientists now ask themselves, how is this person or group going to spend their money?
The answer can be given by thoroughly analyzing the group.\par

An exemplary procedure would be to start by classifying the Internet users according to some pattern. 
This entails agreeing on a unique identifier for a group of users, such as the their country of origin. 
The next step would be to observe this group's behavior. 
For example, one could be interested in the number of times a person accesses a certain website per day. One can regress this behavior in 
data structures attributed to artificial intelligence. Now, given a user of unknown origin and their frequency 
of accessing a certain website, we can draw conclusions about the user's country of origin. 

The utility of this example may be questionable due to simplicity of the experiment. However, similar 
principles can be applied to human behavior on the web. The more one knows about a person or a group, the greater the ability to influence that entity.

This paper is divided into four major parts: relevant literature, technical background, introduction of TwitchTV (in the following simply 'Twitch'), and the critical analysis of the platform.
Within the section on relevant literature we cover some of the resources, mostly available on the web, that we consider suited for further reading on the topic of data collection and data analysis. 
The technical background section delivers a basic understanding of how user data can be collected and explains the most common tracking practices. 
In addition to that, a brief overview of computational intelligence is given to gain insight into how conclusions can be drawn from data. 
The introduction of the Twitch services covers the main aspects of the streaming platform, focusing on its business model and its implications to modern society.  
The analysis section concentrates on the data flow around the Twitch platform and the way Twitch interacts with its customers and partners. 
At this point, we conduct a thorough analysis of the Twitch's data and cookie policy and introduce the findings of our experimental proceedings. 
Finally, we sum up our results in a conclusion while giving a brief overview of possible future work. 
