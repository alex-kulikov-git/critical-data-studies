\section{Twitch}
Twitch is an online streaming platform specializing in video games. 
Twitch is solely the name of the streaming service and should not be confused with the company it is owned by --- namely Twitch Interactive. Another fact we consider significant to keep in mind throughout the course of this work is that Twitch Interactive is a direct subsidiary of Amazon.com, Inc. 

\subsection{Platform}
In a study conducted by the company Lifecourse Associates\footnote{Lifecourse Associates accessible under \url{https://www.lifecourse.com}. }, Twitch is said to be responsible for the fourth-highest U.S. peak Internet traffic, right after Netflix, Google and Apple with their respective streaming services~\cite{twitch-survey}. Besides, while there are no hard facts provided on the matter, it seems sensible to assume that Twitch is about as popular in European countries as it is in the U.S. In total, the platform accounted for 355bn minutes of watched streams from all over the world in 2017~\cite{twitch-minutes-watched}. 

One can differentiate between three different groups using the service: 
\begin{itemize}
	\item Broadcasters, who produce video content. Twitch provides broadcasters with the needed technology to stream games, professionally produced e-sports events or even outdoor activities.
	\item Users, who consume the content by watching streams, videos, setting preferences and participating in integrated online live chats. 
	\item Developers, who embed the streaming platform into their own applications as well as produce extensions to make the video content more interesting and interactive. 
\end{itemize}
Naturally, the platform is an ideal location for brand marketing and product placement due to the high visitor numbers. 

Besides actively watching streams and participating in chat, Twitch offers some convenience functionality to its users. Viewers can follow a specific channel they like in order to receive information and updates on the selected channel. These updates are usually issued in form of an e-mail to the followers, hence requiring their respective e-mail-address. In addition to that, users can subscribe to a channel, which results in them getting access to customized emotes that can be used in the accompanying live chat of the stream and allows them to participate in a 'subscriber only' chat whenever the mode is switched on. Subscriptions are acquired on a monthly basis. Twitch also offers many other features, including various user settings and preferences. However, we restrain from listing all of them in this paper, as we are mainly interested in the presented core functionality for our research purposes. 

\subsection{Business Model}
Twitch generally has two sources of income: the revenue they share with the streamers and the money they make through contracts with third parties. 
The shared revenue consists of subscriptions and donations provided by the users. The previously explained subscription functionality is available after a \$4.99 fee\footnote{\$4.99 at the time of research, last access on July 2, 2018} is paid. Additionally, in 2017 the platform added two additional subscription tiers for \$9.99 and \$24.99 respectively~\cite{twitch-subscriptions}. Users who simultaneously are \textit{Amazon Prime} customers, receive  the option to subscribe to one channel for free once a month. This service was created to establish a stronger connection between Twitch and Amazon users. Twitch shares the money made by subscriptions with the respective broadcaster on a 50-50 ratio. Donations are made by viewers who specifically like a certain broadcaster and therefore want to support them by paying an arbitrary amount of money to them. Donations are also shared on a 50-50 ratio between Twitch and the broadcaster. 
The money Twitch makes through contracts with third parties mostly consists of advertising fees, which the streaming service receives for providing a convenient advertising platform to its partners, such as \textit{Amazon} or \textit{MSI} for instance. 

\subsection{Implications to society}
Another metric that many of the modern companies consider extremely valuable is the number of users their platform               
can reach. In this regard, Twitch has been growing very rapidly over the last several years, thus attracting ever growing numbers of users~\cite{twitch-survey}. 
However, unlike most social media outlets, Twitch does not provide any significant political grasp. It is in the first place motivated by making money, as most businesses are, and thus mainly interested in exposing its audience to the ads or services of its partners. Twitch's viewership is mostly comprised of young people --- teens to young adults\footnote{According to ~\cite{twitch-survey}, "Nearly half of Twitch's visitors are 18- to 34-year-olds."}. These users provide an audience, which has a common interest in computers, games and technology, to Twitch's advertisement partners. Because the viewer base is so young, they are still in the process of forming shopping habits and trusting certain companies. This makes Twitch even more attractive to businesses cooperating with it. Therefore, an argument could be made, that Twitch forms a connection between playing video games and buying advertised products, thus bringing together the two 'worlds' of spending money and leisure. Because the platform accompanies teens from a young age until they are grown adults, behavioral patterns are likely to be formed in their subconsciousness. This practice can be compared to \textit{FIFA} providing companies like \textit{Adidas} or \textit{Nike} with a platform  to advertise their products. While social and behavioral implications of this sort should be kept in mind when discussing how Twitch treats its users' data, they are not directly subject of this paper. As the data is supposedly collected anonymously, there is no direct harm in adjusting advertisements to the field of interests of the customer. In how far this anonymity holds, we will discuss in our analysis of the platform~\sref{analysis}. 
