\section{Conclusion/Outlook}
In conclusion, Twitch is tracking its users just as every major internet player does.
The extent of the tracking does by no means exceed the basic mechanisms that informed users have come to expect from using corporate services.
During our research, we could not find evidence of any violation of what is stated in the privacy policy, which itself is for the most part comprehensive and concise.
The Twitch privacy policy also offers hints on how to opt-out of tracking on their platform,
which revolves around individual opt-outs from all third-party tracking providers Twitch is using.
The feasibility of this approach is highly questionable, since the providers might change arbitrarily without prior notice.
Furthermore, opt-out processes are not standardized or streamlined in any way, such that the effort of doing so for multiple services is hard to estimate.
An early effort, Do-Not-Track (DNT) requests are widely disregarded because there is no standard interpretation and no legal liability connected to them.
Flat out disabling cookies, and therefore making tracking impossible, is no working solution, since Twitch becomes unusable in such a configuration.
Particularly with the introduction of the General Data Protection Regulation (GDPR) in Europe,
this may change in the future since consent to tracking is not necessary for Twitch to serve streams (lawful basis for data usage).
Since opting out is no feasible solution, the only thing left to do for guaranteed protection is to not use the service at all.
Todays online services are so closely interwoven with tracking partners and with user data fueling whole businesses that
the act of using the Internet already means consenting to being tracked and user data to be utilized.
This imposes an array of ethical and practical questions which have to be discussed by modern society as a whole,
the GDPR being the most recent political process. We strongly expect many more discussions on data privacy to arise in the future than it has been the case so far.
Currently, the most viable compromise between usability and privacy seems to be to only register and login to services when it is deemed absolutely necessary, and,
more importantly, to retain cookies for the current session only (usually this would mean until closing the browser).
This obviously means that logins do not persist across sessions, this represents the trade-off for not being trackable across sessions.
